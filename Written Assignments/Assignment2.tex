\documentclass{article}

\usepackage{booktabs} 
\usepackage[english]{babel}
\usepackage{graphicx}
\usepackage{amsmath}

\title {Assignment 2}
\date{}
\author{Wendy Nieuwkamer}

\begin{document}

\maketitle

\section{Question 1}
This question is about \textit{vectorization},i.e. writing expressions
in matrix-vector form. The goal is to vectorize the update rule
for multivariate linear regression.\\

Let $\theta$ be the parameter vector $\theta = \begin{pmatrix} \theta_0 & \theta_1 & \cdots & \theta_n \\ \end{pmatrix} ^T$ and let the i-th data vector be: $ x^{(i)} = \begin{pmatrix} x_0 & x_1 & \cdots & x_n \\\end{pmatrix}^T$ where $ x_0 = 1$.

\subsection{Write the hypothesis function $h_\theta(x)$ as a vectorial expression.}

The summation notation for the hypothesis function is:

\begin{equation*}
	h_\theta (x^{(i)}) = \theta_0 x_0^{(i)} + \theta_1 x_1^{(i)}+ \theta_2 x_2^{(i)}+ ... + \theta_n x_n^{(i)}
\end{equation*}\\

This is the same as the result of the following matrix multiplication:
\begin{equation}
	h_\theta (x^{(i)}) = \theta^T x^{(i)},
\end{equation}

which is a vectorial expression.

\subsection{What is the vectorized expression for the cost function: $J(\theta)$?}
The cost function in the notation used up to now:

\begin{equation*}
		J(\theta) = \frac{1}{2m} \sum\limits_{i=1}^m (h _{\theta} (x ^{(i)}) - y ^{(i)})^2.
\end{equation*}
If we simply insert the vectorial notation of the hypothesis function from last question (1) we get:

\begin{equation*}
		J(\theta) = \frac{1}{2m} \sum\limits_{i=1}^m (\theta^T x^{(i)} - y ^{(i)})^2.
\end{equation*}

\subsection{What is the vectorized expression for the gradient of the cost function?}
\textit{i.e. what is:}

\begin{equation*}
\frac{\delta J(\theta)}{\delta \theta} = \begin{pmatrix} \frac{\delta J(\theta)}{\delta \theta_0} \\ \vdots \\ \frac{\delta J(\theta)}{\delta \theta_n} \end{pmatrix}
\end{equation*}
\textit{Again the explicit summation over the data vectors from the
learning set is allowed here.}\\

The notation we used up until now is:

\begin{equation*}
\frac{\delta J(\theta)}{\delta \theta_j}= \frac{1}{m} \sum\limits_{i=1}^m (h _{\theta} (x ^{(i)}) - y ^{(i)})x_j^{(i)}.
\end{equation*}

If we integrate the vectorized notation of the hypothesis (1) again we get:

\begin{equation*}
\frac{\delta J(\theta)}{\delta \theta_j}= \frac{1}{m} \sum\limits_{i=1}^m ( \theta^T x^{(i)}  - y ^{(i)})x_j^{(i)}.
\end{equation*}

Substituting this summation for the $\delta$ notation will give us the following vector:

\begin{equation}
\frac{\delta J(\theta)}{\delta \theta} = \frac{1}{m}\begin{pmatrix}	\sum\limits_{i=1}^m ( \theta^T x^{(i)}  - y ^{(i)})\\ 
									     		\sum\limits_{i=1}^m ( \theta^T x^{(i)}  - y ^{(i)})x_1^{(i)} \\ 
										      	 \vdots \\
										 	\sum\limits_{i=1}^m ( \theta^T x^{(i)}  - y ^{(i)})x_n^{(i)}
						\end{pmatrix}
\end{equation}

\subsection{What is the vectorized expression for the $\theta$ update rule in the gradient descent procedure?}
The original notation for the update rule for one theta was:

\begin{equation*}
\theta_j = \theta_j - \alpha \frac{1}{m}  \sum\limits_{i=1}^m (h _{\theta} (x ^{(i)}) - y ^{(i)})x_i^{(i)}
\end{equation*}

By writing it in vector notation we update the entire theta instead of each element separately. Using the formulas from (1) and (2) we get:

\begin{equation*}
\theta = \theta - \alpha \frac{1}{m}\begin{pmatrix}	\sum\limits_{i=1}^m ( \theta^T x^{(i)}  - y ^{(i)})\\ 
									     		\sum\limits_{i=1}^m ( \theta^T x^{(i)}  - y ^{(i)})x_1^{(i)} \\ 
										      	 \vdots \\
										 	\sum\limits_{i=1}^m ( \theta^T x^{(i)}  - y ^{(i)})x_n^{(i)}
						\end{pmatrix}
\end{equation*}

\subsection{(BONUS) Remove the explicit summation by using a matrix vector multiplication}
We start by defining the data matrix $X$; every row of $X$ is a training example, the first column containing $x_0^{(1)}, x_0^{(2)}, ..., x_0^{(n)}$.\\

\begin{equation*}
X = \begin{pmatrix} 
	x_0^{(1)} & x_1^{(1)} & \cdots & x_n^{(1)} \\
	x_0^{(2)} & x_1^{(2)} & \cdots & x_n^{(2)} \\
	\vdots & \vdots & & \vdots \\
	x_0^{(m)} & x_1^{(m)} & \cdots & x_n^{(m)}
     \end{pmatrix}
\end{equation*}

The hypothesis function will then be $h_\theta(X) = X \theta$, which results in the matrix:

\begin{equation*}
h_\theta(X) = X \theta =  \begin{pmatrix} 
	x_0^{(1)} & x_1^{(1)} & \cdots & x_n^{(1)} \\
	x_0^{(2)} & x_1^{(2)} & \cdots & x_n^{(2)} \\
	\vdots & \vdots & & \vdots \\
	x_0^{(m)} & x_1^{(m)} & \cdots & x_n^{(m)}
     \end{pmatrix}
\begin{pmatrix}
\theta_0 \\
\theta_1 \\
\vdots \\
\theta_n
\end{pmatrix}
 = \begin{pmatrix}
\theta_0 x_0^{(1)} + \theta_1 x_1^{(1)}+  ... + \theta_n x_n^{(1)} \\
\theta_0 x_0^{(2)} + \theta_1 x_1^{(2)}+  ... + \theta_n x_n^{(2)}\\
\vdots \\
\theta_0 x_0^{(m)} + \theta_1 x_1^{(m)}+ ... + \theta_n x_n^{(m)}
\end{pmatrix}
\end{equation*}

Let Y be the matrix $Y = \begin{pmatrix} y^{(1)} & y^{(2)} & \cdots & y^{(m)} \end{pmatrix} ^T$. Then we can write the derivative of $J(\theta)$ as follows:

\begin{align*}
\frac{\delta J(\theta)}{\delta \theta} &= \frac{1}{m} X(X\theta - Y) \\
&= \frac{1}{m}\begin{pmatrix} 
	x_0^{(1)} & x_1^{(1)} & \cdots & x_n^{(1)} \\
	x_0^{(2)} & x_1^{(2)} & \cdots & x_n^{(2)} \\
	\vdots & \vdots & & \vdots \\
	x_0^{(m)} & x_1^{(m)} & \cdots & x_n^{(m)}
     \end{pmatrix}
\left( 
 \begin{pmatrix}
\theta_0 x_0^{(1)} + \theta_1 x_1^{(1)}+  ... + \theta_n x_n^{(1)} \\
\theta_0 x_0^{(2)} + \theta_1 x_1^{(2)}+  ... + \theta_n x_n^{(2)}\\
\vdots \\
\theta_0 x_0^{(m)} + \theta_1 x_1^{(m)}+ ... + \theta_n x_n^{(m)}
\end{pmatrix}
- \begin{pmatrix} y^{(1)} \\ y^{(2)} \\ \vdots \\ y^{(m)} \end{pmatrix}
\right)
\end{align*}

Thus, the final update rule would be:

\begin{equation*}
\theta = \theta - \alpha \frac{1}{m} X(X\theta - Y)
\end{equation*}

\pagebreak

%\section{Question 2}
%\textit{Consider events with two binary outcomes, X and Y. We encode the two values as 0 and 1. We can represent
%the outcomes of an experiment in a frequency table:}
%
%\begin{table}[h!]
%\centering
%\begin{tabular}{c c c c}
% x & y & freq & P(X=x, Y=y)\\
%\hline
%0&0&a&\\
%0&1&c&\\
%1&0&b&\\
%1&1&d&
%\end{tabular}
%\end{table}
% 
%\subsection{Complete the table by estimating $P(X=x,Y=y)$ for every combination}
%
%
%\begin{table}[h!]
%\centering
%\begin{tabular}{c c c c}
% x & y & freq & P(X=x, Y=y)\\
%\hline
%0&0&a& $\frac{a}{a+b+c+d}$\\
%0&1&c& $\frac{c}{a+b+c+d}$\\
%1&0&b& $\frac{b}{a+b+c+d}$\\
%1&1&d& $\frac{d}{a+b+c+d}$
%\end{tabular}
%\end{table}
%
%
%\subsection{Calculate $P(X=0)$}
%$X=0$ for both the combinations $ X=0, Y=0$, and $X=0, Y=1$. Thus, we are looking at $a$ and $c$. Then, according to the standard formula:
%
%\begin{equation*}
%P(X=0) = \frac{a + c}{a+b+c+d}
%\end{equation*}
%
%\subsection{Calculate $P(X=1|Y=0)$}
%
%\begin{equation*}
%P(X=1|Y=0) = \frac{b}{\frac{a+b}{a+b+c+d}}
%\end{equation*}
%
%\subsection{Calculate $P(X=1 \cup Y=0)$}


\end{document}