\documentclass[12pt, a4paper]{article}

\usepackage{booktabs} 
\usepackage[english]{babel}
\usepackage{graphicx}
\usepackage{amsmath}
\usepackage{hyperref}

\title {Assignment 1}
\date{16-9-2016}
\author{Wendy Nieuwkamer}

\begin{document}

\maketitle

\section{Question 1}
\textit{Suppose that we have historical data of result of soccer matches of teams playing against Ajax. We wantto
use this information to learn to predict at a certain moment whether a team will win, lose or draw against
Ajax. Our approach will be based on Machine Learning.}

	\subsection{Define the given and the goal of the prediction task.
	Classify the learning task.}
	
	Given are the results of soccer matches, the goal is to decide whether a team will win, lose or draw when playing. 
	The learning task is supervised, as we are given inputs with their wanted result. It is a classification problem as 
	every input should be classified as one of three possibilities.
	
	
	\subsection{What would be the form of training data for the learning task? Give a small training set.}
	An example of a training set could be the following table:
	
	\begin{table}
	\centering
	\begin{tabular}{c|c}
	   team & results \\
	   \hline
	   1 & win\\
	   2 & draw\\
	   3 & lose\\
	   1 & lose\\ 
	   3 & draw\\
	   4 & win\\
	\end{tabular}
	\caption{training examples}
	\end{table}

\pagebreak

\section{Question 2}
Use the training data in the table to answer all the subquestions. 

\begin{table}[h!]
\centering
\begin{tabular}{c|c}
   x & y \\
   \hline
   3 & 6\\
   5 & 7\\
   6 & 10
\end{tabular}
\end{table}	

	\subsection{Manually  calculate two iterations of the gradient descent algorithm
	for univariate linear regression function. }
	
	\textit{Initialize the parameters such that the regression function
	passes through the origin (0, 0) and has an angle of 45 degrees. Use a learning rate of 0.1. Give the
	intermediate results of your calculations and also compute the mean-squared error of the function
	after 2 iterations.}\\
	
	\textbf{Gradient Descent Algorithm}\\
	
	Repeat until convergence \{
	\begin{align*}
	\theta _0 &:= \theta _0 - \alpha \frac{1}{m} \sum\limits_{i=1}^m (h _{\theta} (x ^{(i)}) - y ^{(i)})\\
	\theta _1 &:= \theta _1 - \alpha \frac{1}{m} \sum\limits_{i=1}^m (h _{\theta} (x ^{(i)}) - y ^{(i)})x^{(i)} \\
	\end{align*}
	\}\\
	
	The hypothesis $h_\theta$ is:
	\begin{equation*}
	h_\theta (x^{(i)}) = \theta_0 = \theta_1 x^{(i)}
	\end{equation*}\\
	
	For this exercise the learning rate $\alpha=0.1$ and the amount of learning examples $m=3$ are given. The line that passes through the origin and has an angle of 45 degrees relative to 		the x-axis is $f(x) = x$, thus our starting values are $\theta_0 = 0$ and $\theta_1 = 1$.
	
	\paragraph{First Iteration}
	First, calculate the new values for $\theta_0$ and $\theta_1$.
	
		\begin{align*}
		\theta_0 &= 0 - 0.1 \cdot \frac{1}{3}\sum\limits_{i=1}^3 (x ^{(i)} - y ^{(i)})\\
		&= -\frac{1}{30} (-3 + -2 + -4)\\
		&= 0.3 \\
		\\
		\theta_1&=1-0.1 \cdot \frac{1}{3}\sum\limits_{i=1}^3 (x ^{(i)} - y ^{(i)})x^{(i)}\\
		&=1-\frac{1}{30}(-9+-10+-24)\\
		&=1+\frac{43}{30}=\frac{73}{30}=2.43
		\end{align*}
		
		Then, update $\theta_0$ and $\theta_1$.
		
		\begin{align*}
		\theta_0 &:= 0.3 \\
		\theta_1&:=2.43
		\end{align*}
	
	\paragraph{Second Iteration}
	The values for $m$ and $\alpha$ are still the same, but now we use the new values for $\theta_0$ and $\theta_1$. Again, first the calculations.
	
		\begin{align*}
		\theta_0 &= 0.30 - 0.1 \cdot \frac{1}{3}\sum\limits_{i=1}^3 (0.30+2.34x ^{(i)} - y ^{(i)})\\
		&= 0.30-\frac{1}{30} (1.6+5.47+4.9)\\
		&= 0.30 - 0.40 = -0.10\\
		\\
		\theta_1&=2.43-0.1 \cdot \frac{1}{3}\sum\limits_{i=1}^3 (0.30+2.34x ^{(i)} - y ^{(i)})x^{(i)}\\
		&=2.43-\frac{1}{30}(4.80+27.33+29.40)\\
		&=2.43-2.05=0.38
		\end{align*}
		
		\pagebreak
		Then, update $\theta_0$ and $\theta_1$.
		
		\begin{align*}
		\theta_0 &:= -0.10\\
		\theta_1&:=0.38
		\end{align*}
	
	\paragraph{Mean-Squared Error}
	Now we can calculate the mean-squared error to see how good our hypothesis $h_\theta(x^{(i)}) = -0.10 + 0.38x^{(i)}$ is. The formula we use is:
	
		\begin{equation*}
		J(\theta_0, \theta_1) = \frac{1}{2m} \sum\limits_{i=1}^m (h _{\theta} (x ^{(i)}) - y ^{(i)})^2
		\end{equation*}
		
		This function is also called the cost function. We can now insert our values $m=3$, $\theta_0=-0.10$ and $\theta_1=0.38$.
		
		\begin{align*}
		J(\theta_0, \theta_1) &= \frac{1}{6} \sum\limits_{i=1}^3 (0.38x ^{(i)} - 0.10 - y ^{(i)})^2\\
		 &= \frac{1}{6} (24.60+27.04+61.15)\\
		&=18.80
		\end{align*}
		
		Thus, the mean-squared error for $m=3$, $\theta_0=-0.10$, $\theta_1=0.38$ and the given training examples is 18.80.
	
	\subsection{Convert the data to z-scores (with mean = 0, sd = 1), repeat the calculations above and compare the results.}
	\textit{At first I interpreted this question as:} "To calculate the z-scores use mean $\mu=0$ and standard deviation $\sigma=1$."
	\textit{Which leads to the following answer:} "The formula for calculating a z-score is:
	
	\begin{equation*}
	Z = \frac{X-\mu}{\sigma}
	\end{equation*}
	
	Where $X$ is the input value of the training example. Inserting our $\mu$ and $\sigma$ into this formula gives:
	
	\begin{equation*}
	Z = \frac{X-0}{1} = X
	\end{equation*}
	
	Redoing the calculations would not be very useful as with this $\mu$ and $\sigma$ the Z-score would be the same as the original input. It would be more useful to choose $\mu=4.5$ 
	and $\sigma=1.125$ for $x$ as this actually changes the values."\\

	\textit{Then I realized that the Z-scores were supposed to have mean $\mu=0$ and standard deviation $\sigma=1$, which leads to the following answer:}
	First, we have to calculate the mean $\mu$ of our training examples. The formula for this is:

\begin{equation*}
\mu = \frac{\sum\limits_{i=1}^m x^{(i)}}{m}
\end{equation*}

Where $m$ is the amount of training examples. So, for our $x$ values this gives $\mu_x = 4\frac{2}{3} = 4.67$ and for the $y$ values it gives $\mu_y = 7.67$.

Now I ran into some questions. One of them is when you apply this to your $x$ values, do you have to do the same with your $y$ values. And why?
Also, when we calculate the standard deviation, do we calculate the Population standard deviation or the Sample standard deviation?

I decided to take the Population Standard deviation as calculated by the website \url{https://www.mathsisfun.com/data/standard-deviation-calculator.html}. 
The reason for this is that we were asked to use $\sigma$, which is only used for the population standard deviation.
This gave me the value $\sigma_x = 1.247219129$ and $\sigma_y = 1.699673171$. Now applying the formula above to my values gives me the following:

\begin{table}[h!]
\centering
\begin{tabular}{c|c}
   x & y \\
   \hline
   -1.34 & -1.00\\
   0.27 & -0.39\\
   1.07 & 1.37
\end{tabular}
\end{table}

With these new values we repeat the same motions as in the last subsection. Again, our starting values are $\alpha = 0.1$, $\mu = 3$, $\theta_0 = 0$ and $\theta_1=1$.

\pagebreak

\paragraph{First Iteration}
	First, calculate the new values for $\theta_0$ and $\theta_1$.
	
		\begin{align*}
		\theta_0 &= 0 - 0.1 \cdot \frac{1}{3}\sum\limits_{i=1}^3 (x ^{(i)} - y ^{(i)})\\
		&= -\frac{1}{30} (-0.34+0.66-0.30)\\
		&=- 6.67e^{-4} \\
		\\
		\theta_1&=1-0.1 \cdot \frac{1}{3}\sum\limits_{i=1}^3 (x ^{(i)} - y ^{(i)})x^{(i)}\\
		&=1-\frac{1}{30}(0.46+0.18-0.32)\\
		&=1+0.01=0.99
		\end{align*}
		
		Then, update $\theta_0$ and $\theta_1$.
		
		\begin{align*}
		\theta_0 &:= -6.67e^{-4}\\
		\theta_1&:=0.99
		\end{align*}

I don't have time to do the second iteration and calculate the mean-squared error for the results as I did not understand the question in time.
	

\pagebreak


\section{Question 4}
\textit{Derive an equation that can be used to find the optimal value of the parameter $\theta_1$ for
univariate linear regression without doing gradient descent. This can be done by setting the value of the
derivative equal to 0. You may assume that the value of $\theta_0$ is fixed.}

	To find a formula for $\theta_1$ we take zero as the value of the derivative relative to $\theta_1$ of the cost function: 
	
	\begin{equation*}
	\frac{\delta}{\delta\theta_1}J(\theta_0, \theta_1) = 0
	\end{equation*}
	
	We have seen this derivative before as a part of the gradient descent algorithm, so the next step is easy. 
	
	\begin{equation*}
	 \frac{1}{m} \sum\limits_{i=1}^m (h _{\theta} (x ^{(i)}) - y ^{(i)})x^{(i)}) = 0
	\end{equation*}
	
	Now, in order to simplify the equation, we first multiply both sides by $m$ (1), write out the hypothesis $h_\theta$ (2) and write out the multiplication with $x^{(i)}$ (3).
	
	\begin{align}
	&\sum\limits_{i=1}^m(h_\theta(x^{(i)})-y^{(i)})x^{(i)}) &= 0\\
	&\sum\limits_{i=1}^m((\theta_0+\theta_1x^{(i)} - y^{(i)})x^{(i)}) &= 0\\
	&\sum\limits_{i=1}^m(\theta_0x^{(i)} + \theta_1(x^{(i)})^2 - y^{(i)}x^{(i)}) &= 0
	\end{align}
	
	Now, we can breakup the summation (4), factor out $\theta_1$ (5) and the summation without $\theta_1$ to the right side of the equation (6).
	
	\begin{align}
	\sum\limits_{i=1}^m(\theta_1(x^{(i)})^2 ) + \sum\limits_{i=1}^m(\theta_0x^{(i)} - y^{(i)}x^{(i)}) &= 0\\
	\theta_1\sum\limits_{i=1}^m((x^{(i)})^2 ) + \sum\limits_{i=1}^m(\theta_0x^{(i)} - y^{(i)}x^{(i)}) &= 0\\
	\theta_1\sum\limits_{i=1}^m((x^{(i)})^2 )   &= - \sum\limits_{i=1}^m(\theta_0x^{(i)} - y^{(i)}x^{(i)})
	\end{align}
	
	Finally, divide both sides by the left sum to get an equation for $\theta_1$.
	
	\begin{equation*}
	\theta_1 = -\frac{\sum\limits_{i=1}^m(\theta_0x^{(i)} - y^{(i)}x^{(i)})}{\sum\limits_{i=1}^m((x^{(i)})^2 )}
	\end{equation*}













\end{document}